%% Soft robots are difficult to model, but safe to observe
Soft robots are challenging to model due to their nonlinear behavior.
However, their soft bodies make it possible to safely observe their behavior under random control inputs, making them amenable to large-scale data collection and system identification.
%% Nonlinear system identification is hard, linear sysid is easy
Nonlinear system identification methods generally consist of solving optimization problems with limited convergence guarantees, therefore the resulting model may be suboptimal \David{the solution is suboptimal, the model is what? incorrect? Not the best possible description? Not sure}.
Linear system identification, i.e. linear regression, does not suffer from this shortcoming, but linear models have shown limited ability to capture soft robot dynamics.
\David{In general I am not a fan of the structure: ``You could do A and it fails, or you could B and it fails, so here let's do C''.  Instead, I would say ``we use C to this, which doesn't suffer from the drawbacks of A and B'', C being, of course the koopman idea.  I.e., something along the lines of: ``In this paper, we propose such a sys-id based on the Koopman operator theory. This theory offers a way ...  The approach thus doesn't suffer from ....''}
%% Koopman operator theory
Koopman operator theory offers a way to represent nonlinear systems as linear systems in the infinite dimensional space of observables.
This enables a model of a nonlinear system to be constructed via linear regression of observed data.
%% Contribution
In this paper, a system identification method based on Koopman operator theory is used to construct a dynamic model of a pneumatically actuated soft robot arm.
This model is shown to capture the dynamic behavior of the real system \Ram{many real systems hopefully?} better than several other  state-of-the-art  nonlinear  system identification  methods  including  a  neural  network, NARMAX, nonlinear Hammerstein-Wiener, and state space model.
\David{Can you add some quantitative results to the abstract?}