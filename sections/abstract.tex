%% Soft robots are difficult to model, but safe to observe
Soft robots are challenging to model due to their \sout{highly} \Ram{what does highly mean?} nonlinear behavior.
However, their soft bodies make it possible to safely observe their behavior under random control inputs, making them amenable to large-scale data collection \hl{and system identification}.
%% Nonlinear system identification is hard, linear sysid is easy
\Ram{This is a sharp transition between ideas.}
Nonlinear system identification methods generally consist of solving optimization problems with limited convergence guarantees, therefore the resulting model may be suboptimal.
Linear system identification, i.e. linear regression, does not suffer from this shortcoming, \hl{but linear models have shown limited ability to capture soft robot dynamics} \Ram{but linear systems have shown limited ability to represent soft robot dynamics??}.
%% Koopman operator theory
Koopman operator theory offers a way to represent nonlinear systems as linear systems in the infinite dimensional space of observables.
This enables a model of a nonlinear system to be constructed via linear regression of observed data.
%% Contribution
\hl{In this paper, a system identification method based on Koopman operator theory is used to construct a dynamic model of a pneumatically actuated soft robot arm}
\sout{In this paper we apply a system identification method based on Koopman operator theory to identify a dynamic model of a pneumatically actuated soft robot arm} \Ram{abstracts should be written in the third person.}.
This model is shown to capture the dynamic behavior of the real system better than several other  state-of-the-art  nonlinear  system identification  methods  including  a  neural  network, NARMAX, nonlinear Hammerstein-Wiener, and state space model \Ram{try to emphasize that we've done it on a real robot even more loudly.}.