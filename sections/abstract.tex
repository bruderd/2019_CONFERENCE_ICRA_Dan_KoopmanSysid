%% Soft robots are difficult to model, but safe to observe
Soft robots are challenging to model due to their nonlinear behavior.
However, their soft bodies make it possible to safely observe their behavior under random control inputs, making them amenable to large-scale data collection and system identification.
%%
This paper implements and evaluates a system identification method based on Koopman operator theory.
This theory offers a way to represent a nonlinear system as a linear system in the infinite-dimensional space of real-valued functions called observables, enabling models of nonlinear systems to be constructed via linear regression of observed data.
%%
% The approach does not suffer from the limited convergence guarantees of other nonlinear system identification methods, which typically consist of solving a nonlinear non-convex optimization problem. 
The approach does not suffer from some of the shortcomings of other nonlinear system identification methods, which typically require the manual tuning of training parameters and have limited convergence guarantees.
%%
A dynamic model of a pneumatic soft robot arm is constructed via this method, and used to predict the behavior of the real system.
The total normalized-root-mean-square error (NRMSE) of its predictions over twelve validation trials is lower than
that of several other identified models including a neural network, NLARX, nonlinear Hammerstein-Wiener, and linear state space model.
% The total normalized-root-mean-square error (NRMSE) of its predictions over twelve validation trials was 2.1\%.
% In comparison, the NRMSE of the predictions of several other identified models including a neural network, NLARX, nonlinear Hammerstein-Wiener, and linear state space model were greater than 4.4\%.








%% OLD TEXT BELOW THIS LINE 

% \David{In general I am not a fan of the structure: ``You could do A and it fails, or you could B and it fails, so here let's do C''.  Instead, I would say ``we use C to this, which doesn't suffer from the drawbacks of A and B'', C being, of course the koopman idea.  I.e., something along the lines of: ``In this paper, we propose such a sys-id based on the Koopman operator theory. This theory offers a way ...  The approach thus doesn't suffer from ....''}

% %% Nonlinear system identification is hard, linear sysid is easy
% Nonlinear system identification methods generally consist of solving optimization problems with limited convergence guarantees, therefore the resulting model may be suboptimal \David{the solution is suboptimal, the model is what? incorrect? Not the best possible description? Not sure}.
% Linear system identification, i.e. linear regression, does not suffer from this shortcoming, but linear models have shown limited ability to capture soft robot dynamics.

% %% Koopman operator theory
% Koopman operator theory offers a way to represent nonlinear systems as linear systems in the infinite dimensional space of observables.
% This enables a model of a nonlinear system to be constructed via linear regression of observed data.
% %% Contribution
% In this paper, a system identification method based on Koopman operator theory is used to construct a dynamic model of a pneumatically actuated soft robot arm.