\section{Discussion and Conclusion}
\label{sec:conclusion}

%% Summary of what we've shown
We have successfully applied a system identification technique based on Koopman operator theory to a soft robot and shown that the model generated outperforms those generated by several other state-of-the-art nonlinear system identification methods.
% This data-driven approach enables the identification of dynamic models of soft robots which can be used in control.
Perhaps unsurprisingly, the linear state space model was not able to capture the nonlinear dynamics of the robot as well as the nonlinear Koopman model.
%% Why does Koopman model do better than nonlinear models
As for the nonlinear models, there are several likely reasons why the performance of the Koopman model was superior in this case as well.
Since the Koopman model is a state space model, simulations can be initialized from the same initial condition as the real system.
This is not the case for the other learned nonlinear models since they do not have an internal state corresponding to the physical state of the robot.
They act as black-box models only capable of mapping inputs to outputs.
% serve as black-box models only capable of mapping inputs to outputs without an internal state that corresponds to the physical state of the robot.
% For simulations over short time horizons, this effect can contribute significantly to error.

%% Other models rely on manual tuning so success is highly dependent on training parameters chosen
\Dan{our system works just as well in an idiot's hands}
Another reason for the model's success is its lack of tuning parameters.
In the hands of an experienced engineer a neural network model could be generated that outperforms 
The model generated is consistent 


%% Challenges and how it could be improved
\Dan{While promising, more work is needed...}
While this has been successful for modeling the soft robot presented in this work, further investigation is required to determine if the same performance can be readily realized on systems that are more chaotic, higher dimensional, etc...
The method could generate models that capture the behavior of a broader range of systems by changing the type or increasing the size of the finite-dimensional basis.

%% Numerical challenges
% There are still numerical challenges to applying this approach to arbitrary systems.
% As stated previously, the method fails if not enough data points are supplied, but if the training data set is too large, it becomes impractical to implement due to computer memory constraints.
% Therefore, a focus of future work will be to investigate how to generate a representative sampling of system behavior of sufficient size to ensure the method can be successfully implemented.
% This approach also suffers from the curse of dimensionality...


%% Future work / remaining challenges
Soft robots are notoriously difficult to model, but uniquely safe to observe under arbitrary control inputs.
This makes them amenable to large-scale data collection and data-driven modeling methods. 
This paper demonstrates the utility of Koopman operator theory to make accurate nonlinear dynamic models easier to construct, enabling  the  rapid  development  of control strategies that exploit the unique characteristics of soft robots.
Future work will aim to generalize this approach to higher dimensional models, non-polynomial models, and models that account for external loading and contact forces.