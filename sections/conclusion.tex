\section{Conclusion}
\label{sec:conclusion}

%% Summary of what we've shown
We have successfully applied a system identification technique based on Koopman operator theory to a soft robot, and shown that the model generated outperforms those generated by other state-of-the-art nonlinear system identification methods.
Linear system identification of this nonlinear systems is possible by exploiting the fact that nonlinear systems have linear representations in the infinite dimensional space of observables.
Even though in practice it is intractable to identify an infinite-dimensional model, by identifying the projection of the model on a sufficiently high-dimensional subspace of observables, we can recover an approximate model of the nonlinear system that still captures its behavior.

%% Future work / remaining challenges
Soft robots are notoriously difficult to model, but uniquely safe to observe under arbitrary control inputs.
Thus system identification is an ideal choice for modeling them.
This paper demonstrates the potential of Koopman operator theory to improve upon existing system identification methods for soft robots.
Future work will generalize this approach to higher dimensional models, non-polynomial models, and models that account for external loading and contact forces.



% %% Why this is important/significant
% That's not all...
% Koopman system identification is well suited for identification of dynamic models for soft robots
% The potential applications of this lifting technique are far reaching.
% The linear system representation could be leveraged for control



% %% Future possibilities


% Predictors for MPC, observers, simultaneous control and system identification, adaptive control (like an arm that picks something up and immediately updates it's model of itself...)