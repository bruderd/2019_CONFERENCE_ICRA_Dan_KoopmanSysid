\section{Koopman Operator Method for \\ System Identification}
\label{sec:theory}

%% Overview of method
The system identification method utilized in this work exploits the fact that any finite-dimensional nonlinear system has an equivalent infinite-dimensional linear representation in the space of real-valued functions of the system's state and input.
In this space of real-valued functions, the (linear) Koopman operator describes the flow of functions along trajectories of the system.
The relationship between the finite and infinite dimensional representations of the system is bijective and well-defined \cite{lasota2013chaos}.
This enables us to approximate the Koopman operator via linear regression on observed data, then extract the equivalent nonlinear system representation.
% A thorough explanation of this method can be found in \cite{mauroy2016linear} or \cite{mauroy2017koopman}.
The remainder of this section summarizes the system identification method presented in \cite{mauroy2016linear} and \cite{mauroy2017koopman} applied to a system with known input, which is later employed and validated on a real soft robotic system.


%% Overview of lifting technique
\subsection{Overview of Lifting Technique}

%% System representation in state space
Consider a dynamical system
\begin{align}
    \dot{x} &= \Fv(\xv,\uv)    \label{eq:nlsys}
\end{align}
where $\xv \in \Real^n$ is the state of the system, $\uv \in \Real^m$ is the input and ${F}$ is continuously differentiable in $x$.
Denote by $\phi(t,\xv_0,\uv)$ the solution to \eqref{eq:nlsys} at time $t$ when beginning with the initial condition $\xv_0$ at time $0$ and a constant input $\uv$ applied for all time between $0$ and $t$.
For simplicity, we denote this map, which is referred to as the \emph{flow map}, by $\phi^t (\xv_0, \uv)$ instead of $\phi (t, \xv_0, \uv)$.

%% System representation in the space of observables
The system can be lifted to an infinite dimensional function space $\F = L^2(X \times U, \Real)$ where $X \subset \Real^n$ and $U \subset \Real^m$ are compact subsets and $ L^2(X \times U, \Real)$ is the space of square integrable real-valued functions with domain $X \times U$.
Elements of $\F$ are called \emph{observables}.
In $\F$, the flow of the system is characterized by the set %semigroup 
of Koopman operators 
$U^t : \F \to \F$, for each $t \geq 0$,
which describes the evolution of the observables $f \in \F$ along the trajectories of the system according to the following definition:
\begin{align}
    U^t f = f \circ \phi^t      
    % && \forall f \in \F, t \geq 0
    \label{eq:koopman}
\end{align}
As desired, $U^t$ is a linear operator even if the system \eqref{eq:nlsys} is nonlinear, since for $f_1, f_2 \in \F$ and $\lambda_1, \lambda_2 \in \Real$
\begin{align}
    \begin{split}
    U^t (\lambda_1 f_1 + \lambda_2 f_2) &= \lambda_1 f_1 \circ \phi^t + \lambda_2 f_2 \circ \phi^t \\
    &= \lambda_1 U^t f_1 + \lambda_2 U^t f_2.
    \end{split}
\end{align}
Thus the Koopman operator provides a linear representation of the flow of the system in the infinite-dimensional space of observables (see Fig. \ref{fig:overview}) \cite{budivsic2012applied}. 

%% Relationship between representations of the system
One can show that there is a one-to-one correspondence between the infinite-dimensional Koopman operator and finite-dimensional vector field.
To understand this relationship, consider the derivative of an observable, $\dot{f}$, along trajectories of the system:
\begin{align}
    \dot{f}(x,u) &= \frac{\partial f}{\partial x} \frac{dx}{dt} + \frac{\partial f}{\partial u}\frac{du}{dt} \\
           &= \frac{\partial f}{\partial x} F(x,u) \\
           &= ( F \cdot \nabla_{\xv} ) f(x,u)
    \label{eq:fdot}
\end{align}
where the second relation follows since $u$ is held constant and where $\nabla_x$ is the gradient with respect to $x$.
Since this equation holds for all observables, we introduce the infinitesimal generator of the Koopman operator ${A:\F \to \F}$ \cite[Equation 7.6.5]{lasota2013chaos} which is defined in terms of the vector field $F$ as
\begin{align}
    % L &= \Fv \cdot \mtx{ \frac{\partial}{\partial \xv} & \frac{\partial}{\partial \uv} }^T
    A &= \Fv \cdot \nabla_{\xv}
    \label{eq:L2F}
\end{align}
The infinitesimal generator thus describes the dynamics of the observables along trajectories of the system. 
Recalling that the Koopman operator describes the flow of observables, one can show that the Koopman operater is associated with its infinitesimal generator via the following relation:
% where $\nabla_x$ is the gradient with respect to $x$, and $L$ is the infinitesimal generator of the Koopman operator  \cite[Section 7.6]{lasota2013chaos} \Ram{1. you never explain what the infinitestimal generator is...or even cite an appropriate reference. 2. it would help the reader if you described the domain and range of $L$} which satisfies
\begin{align}
    U^t &= e^{A t} = \sum_{k=0}^\infty \frac{t^k}{k!} A^k
    \label{eq:U2L}
\end{align}
In Section \ref{sec:step3} Equations \eqref{eq:L2F} \eqref{eq:U2L} are used to solve for the vector field $F$ when the Koopman operator $U^t$ is known.

%% Steps of sysid method
By providing a linear representation of a system with a one-to-one correspondence to its nonlinear representation, Koopman operator theory enables linear system identification of nonlinear systems. 
This process proceeds in three steps which are summarized by Algorithm \ref{alg:sysid}.
In step one, measured states of the system are lifted to the space of observables.
Step two consists of performing least-squares regression on the lifted data to obtain an approximation of the Koopman operator.
In step three, an approximation of the nonlinear vector field $\Fv$ is obtained via \eqref{eq:L2F} and \eqref{eq:U2L}.
The following three subsections describe each of these steps in more detail.


%% ALGORITHM
\begin{algorithm}
\SetAlgoLined
\KwIn{$ \{ (\xv_k,\uv_k),(\xv_{k+1},\uv_k) \} \text{ for } k = 1, ... , K$}
\textbf{Step 1:} Lift data via \eqref{eq:lift} \\
% \begin{algomathdisplay}
%     \psi(x_k,u_k) \hspace{15pt} \forall k \in 1,...,K
% \end{algomathdisplay} \\
\textbf{Step 2:} Approximate Koopman operator, $\bar{U}^{T_s}$ via \eqref{eq:Ubar} \\
% \begin{algomathdisplay}
%     % \text{Step 2: (Approximate Koopman Operator) }
%     \bar{U}^{T_s} := \Psi_x^\dagger \Psi_y
% \end{algomathdisplay} \\
\textbf{Step 3:} Approximate Vector Field, $\bar{F}$ via \eqref{eq:U2L} and \eqref{eq:Fbar} \\
% \begin{algomathdisplay}
%     \text{(a) } \bar{A} := \frac{1}{T_s} \log \bar{U}^{T_s} \hspace{20pt} \text{(b) } \bar{F} := \frac{\partial \Omega_{x}^\dagger}{\partial x} \underline{\bar{A}}^T \Omega_{x}
% \end{algomathdisplay} \\
\KwOut{$\bar{\Fv}$}
 \caption{Koopman-Based System Identification}
 \label{alg:sysid}
\end{algorithm}


%% STEP 1: Lifting the Data
\subsection{Step 1: Lifting the Data}   \label{sec:step1}

%% Overview of this step
The first step of the Koopman-based system identification method consists of converting empirical data into a form that can be used to identify a linear model in the space of observables.
Theoretically this process would consist of ``lifting'' state measurements into the infinite-dimensional space of observables $\F$.
To be implementable, however, measurements can only be lifted into a finite-dimensional subspace.
%% Approximation must be finite dimensional=
Define $\bar{\F} \subset \F$ to be the subspace of $\F$ spanned by $N$ linearly independent basis functions $\{ \psi_k \}_{k=1}^N$ 
(e.g. monomials, sinusoids, exponentials).
%
Any observable ${\bar{f} \in \bar{\F}}$ can be written as a linear combination of elements of its basis
\begin{align}
    \bar{f} &= \alpha_1 \psi_{1} + \cdots + \alpha_N \psi_N.
\end{align}
Note that the vector of coefficients ${\alpha = \mtx{\alpha_1 & \cdots & \alpha_N}^T}$ provide a \emph{vector representation} for any $\bar{f} \in \bar{\F}$.
To evaluate $\bar{f}$ at a given state $\xv$ and constant input $\uv$, we introduce the \emph{lifting function} ${\psi}:\Real^n \times \Real^m \to \Real^N$ defined as:
\begin{align}
    {\psi}(\xv, \uv) = \mtx{ \psi_1(\xv, \uv) & \cdots & \psi_N(\xv, \uv)}^T
    \label{eq:lift}
\end{align}
Then, $\bar{f}(\xv, \uv)$ can be expressed in vector form as
\begin{align}
    \bar{f}(\xv, \uv) &= {\alpha}^T {\psi}(\xv, \uv).
    \label{eq:fxu}
\end{align}
We refer to ${\psi}(\xv, \uv)$ as an $N$-dimensional ``lifted'' version of ${(\xv, \uv)}$, since multiplying ${\psi}(\xv, \uv)$ by the vector representation of an observable yields the value of the observable at $(\xv, \uv)$.



%% STEP 2: Approximating the Koopman Operator
\subsection{Step 2: Approximating the Koopman Operator} \label{sec:step2}

The second step of the Koopman-based system identification method is to identify the Koopman operator that best describes the flow of the lifted versions of measured data points.
While the Koopman operator is theoretically infinite-dimensional, for practical purposes we identify a finite-dimensional approximation of it in $\bar{\F}$ which we denote by $\bar{U}^t$.
Note, that $\bar{U}^t$ can be represented by an $N \times N$ matrix which operates on observables via matrix multiplication:
\begin{align}
    \bar{U}^t {\alpha} &= {\beta} 
    % && {\alpha}, {\beta} \in \F_N .
    \label{eq:Ubar}
\end{align}
where $\alpha, \beta$ are vector representations of observables in $\bar{\F}$.
Our goal is to find a $\bar{U}^t$ that describes the action of the infinite dimensional Koopman operator $U^t$ as accurately as possible in the $L^2$-norm sense on the finite dimensional subspace $\bar{\F}$  of all observables.
% We want to find $\bar{U}^t$ such that it describes the action of the infinite-dimensional Koopman operator $U^t$ as accurately as possible, i.e.
% \begin{align}
%     U^t f(\xv,\uv) &\approx ( \bar{U}^t {\alpha} )^T {\psi}(\xv, \uv)
% \end{align}
% \hl{for all $f \in \F_n$ with $\alpha$ as its vector representation}.
% From \eqref{eq:koopman}, the Koopman operator describes the flow of observables in $\F$.
Therefore, to perfectly mimic the action of $U^t$ acting on an observable in $\bar{\F} \subset \F$, the following should be true
\begin{align}
    ( \bar{U}^t {\alpha} )^T {\psi}(\xv, \uv) &=
    {\alpha}^T {\psi} \left( \phi^t(\xv,\uv), \uv \right).
    \label{eq:Ubar}
\end{align}
Since this is a linear equation, it follows that for a given ${x \in \Real^n, \uv \in \Real^m}$, solving \eqref{eq:Ubar} for $\bar{U}^t$ yields the best approximation of $U^t$ on $\bar{\F}$ in the $L^2$-norm sense:
\begin{align}
    \bar{U}^t = \left( {\psi}(\xv, \uv)^T \right)^\dagger {\psi}( \phi^t(\xv,\uv), \uv )^T
    \label{eq:Uapprox}
\end{align}
where superscript $\dagger$ denotes the least-squares pseudoinverse.

%% this whole paragraph should be removed...
% \sout{
% From \eqref{eq:koopman}, \eqref{eq:fxu}, and \eqref{eq:Ubar} we can approximate the effect of applying the infinite-dimensional Koopman operator $U^t$ to some $f \in \F_N$ using its finite-dimensional projection
% % \begin{align}
% %     U^t f(\xv,\uv) &\approx ( \bar{U}^t {\alphapeolpe} )^T {\psi}(\xv, \uv) =
% %     {\alpha}^T {\psi} \left( \phi^t(\xv,\uv), \uv \right) \\
% %     \label{eq:Utfapprox}
% % \end{align}
% where $\alpha$ is the vector representation of $f$.
% \Ram{you probably want to explain why each of these equations follow and I really hate when things are written down with approximations like that since with approximations without any explanation as to why its approximate.}
% From this relation the projection of the Koopman operator can be approximated
% % \begin{align}
% %     \bar{U}^t &\approx \left( {\psi}(\xv, \uv)^T \right)^\dagger {\psi}( \phi^t(\xv,\uv), \uv )^T
% %     \label{eq:Uapprox}
% % \end{align}
% where ${\psi}^\dagger$ denotes the pseudoinverse of ${\psi}$ \Ram{Same comment as earlier.}.
% }


%% How this is done on our system
To approximate the Koopman operator from a set of experimental data, we take $K+1$ discrete state measurements with sampling period $T_s$. Under the assumption that the control input is constant between samples, we separate the data into a set of $K$ so-called ``snapshot pairs'' of the form ${ \{(\xv_k,\uv_k),(y_{k},\uv_k)\} \in \Real^{(n \times m) \times 2} }$ where
\begin{align}
    y_{k} &= \phi^{T_s} (x_k, u_k) + \sigma_k
\end{align}
and $\sigma_k$ denotes measurement noise.
For our basis of $\bar{\F}$, we choose the basis of monomials of $\xv$ and $\uv$ with total degree less than or equal to $w$, which implies ${N=(n+m+w)!/\left((n+m)!w!\right)}$ \cite[Section III]{mauroy2016linear}. 
We then lift all of the snapshot pairs according to \eqref{eq:lift} and compile them into the following $K \times N$ matrices:
\begin{align}
    &\Psi_x = \mtx{ {\psi}(\xv_1, \uv_1)^T \\ \vdots \\  {\psi}(\xv_K, \uv_K)^T}
    &&\Psi_y = \mtx{ {\psi}(y_1, \uv_1)^T \\ \vdots \\  {\psi}(y_K, \uv_{K})^T}
\end{align}
 $\bar{U}^{T_s}$ is chosen so that it yields the least squares best fit to all of the observed data, which, following from \eqref{eq:Uapprox}, is given by 
\begin{align}
    \bar{U}^{T_s} &:= \Psi_x^\dagger \Psi_y.
\end{align}
% where ${\Psi}^\dagger$ denotes the least-squares pseudoinverse of ${\Psi}$.


%% STEP 3: Obtaining the vector field
\subsection{Step 3: Obtaining the Vector Field} \label{sec:step3}

The final step of the Koopman-based system identification method is to identify the nonlinear vector field by making use of the one-to-one correspondence between the infinite and finite dimensional system representations.
% Consider again the vector field $\Fv(\xv,\uv)=\mtx{F_1(\xv,\uv) & \cdots & F_n(\xv,\uv)}^T$.
% Each $F_i$ is a function of state and input that can be represented as a linear combination of the basis elements of $\F$.
As earlier, our goal is to find an $\bar{F}$ that describes the behavior of the vector field $F$ as accurately as possible in the $L^2$-norm sense on the finite dimensional subspace $\bar{\F}$.

% Our goal is to find a $\bar{U}^t$ that describes the action of the infinite dimensional Koopman operator $U^t$ as accurately as possible in the $L^2$-norm sense on the finite dimensional subspace $\F_N$  of all observables.

% As we did earlier, we restrict ourselves to represent $F$ using elements from the finite-dimensional subspace $\F_N$. 
% In particular, our goal is to find a finite dimensional representation 

% matrix of coefficients $C \in \Real^{n \times N}$ to 
% % \begin{align}
% %     C &= \mtx{ c_{1,1} & \cdots & c_{1,N} \\ \vdots & \ddots & \vdots \\ c_{n,1} & \cdots & c_{n,N} }
% %     \label{eq:C}
% % \end{align}

% the vector field resulting from this system identification method is only a linear combination of the basis elements of $\F_N$:
% \begin{align}
%     \Fv(\xv, \uv) &\approx C {\psi}(\xv, \uv)
%     \label{eq:F2C}
% \end{align}
% where $C$ is an $n \times N$ matrix of coefficients
% \begin{align}
%     C &= \mtx{ c_{1,1} & \cdots & c_{1,N} \\ \vdots & \ddots & \vdots \\ c_{n,1} & \cdots & c_{n,N} }
%     \label{eq:C}
% \end{align}
% Therefore, identifying the vector field reduces to the task of identifying $nN$ total coefficients $c_{i,j}$.

The vector field is related to the Koopman operator through its infinitesimal generator according to Equation \eqref{eq:L2F}.
% \Ram{sharp transition between the previous paragraph and the ensuing one}
With the approximation of the Koopman operator $\bar{U}^{T_s}$ found in Section \ref{sec:step2}, we can solve for the infinitesimal generator of the set of Koopman operators $\bar{A}$ by inverting \eqref{eq:U2L}:
\begin{align}
    \bar{A} &= \frac{1}{T_s} \log{ \bar{U}^t } \in \Real^{N \times N}
\end{align}
where $\log$ denotes the principal matrix logarithm \cite[Chapter 11]{higham2008functions}.
Note that the principal matrix logarithm is only defined for matrices whose eigenvalues all have non-negative real components, and that $\bar{U}^t$ may have zero or negative eigenvalues when the number of data points is too small \cite{mauroy2016linear}.
Therefore this method might fail if the number of data points is insufficient.
In this instance, more system measurements can be taken to resolve the issue.

With $\bar{A}$ known, \eqref{eq:L2F} can be used to identify $\bar{F}$.
Consider $A$ applied to an observable $f \in \F$.
According to \eqref{eq:L2F}, this is equivalent to the inner product of the vector field $\Fv$ and the gradient of ${f}$ with respect to $x$:
\begin{align}
    A f(\xv,\uv) &= \frac{\partial f(\xv,\uv)}{\partial \xv} \Fv(\xv,\uv).
    \label{eq:L2Fxu}
\end{align}
% \Dan{i.e. the dynamics of the observable $\partial f / \partial t$}
Let $\alpha \in \Real^N$ be the vector representation $\bar{f}$, the projection of ${f}$ onto $\bar{\F}$.
Then from \eqref{eq:fxu} the finite-dimensional equivalent of \eqref{eq:L2Fxu} is given by
\begin{align}
    (\bar{A} {\alpha})^T {\psi}(\xv,\uv) &= {\alpha}^T \frac{\partial \psi(\xv, \uv)}{\partial \xv} \bar{F}.
    \label{eq:Lbar2C}
\end{align}
We seek the vector field $\bar{F}$ such that \eqref{eq:Lbar2C} holds as well as possible in the $L^2$-norm sense for all observed data.
Therefore we choose the least-square solution to \eqref{eq:Lbar2C} over the set of all observed data points $\left\{ (x_k,u_k) | k = 1,...,K \right\}$ which is given by
\begin{align}
    \bar{F} &= \mtx{ \frac{\partial \psi(\xv_1, \uv_1)}{\partial \xv} \\ \vdots \\ \frac{\partial \psi(\xv_K, \uv_K)}{\partial \xv} }^\dagger
    \mtx{ \bar{A}^T & \cdots & 0 \\ \vdots & \ddots & \vdots \\ 0 & \cdots & \bar{A}^T }
    \mtx{ \psi(x_1,u_1) \\ \vdots \\ \psi(x_K,u_K) }.
    \label{eq:Fbar}
\end{align}
% \begin{align}
%     \bar{F} &\approx \mtx{ \frac{\partial \psi(\xv_1, \uv_1)}{\partial \xv} \\ \vdots \\ \frac{\partial \psi(\xv_K, \uv_K)}{\partial \xv} }^\dagger
%         \mtx{ \bar{L}^T \\ \vdots \\ \bar{L}^T }.
% \end{align}
For a more thorough treatment of this process, see \cite{mauroy2016linear, mauroy2017koopman}.